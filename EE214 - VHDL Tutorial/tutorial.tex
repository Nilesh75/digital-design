\documentclass[a4paper]{article}
\usepackage[utf8]{inputenc}
\usepackage[left=1in,right=1in,bottom=0.8in]{geometry}
\usepackage{enumitem}
\usepackage{graphicx}
\pagenumbering{gobble}
\usepackage{hyperref}
\hypersetup{
    colorlinks=true,
    linkcolor=blue,
    filecolor=magenta,      
    urlcolor=blue,
}

\title{An Introduction to VHDL}
\author{\it Dhruv Ilesh Shah}
\date{January 12-13, 2017}

\begin{document}
\maketitle

\begin{center}
{\bf VHDL} $\rightarrow$ VHSIC {\em(Very High Speed Integrated Chip)} Hardware Description Language \\
A great resource/guide would be \href{http://home.gna.org/ghdl/ghdl/index.html}{The GHDL Homepage}
\end{center}
\section*{Important Features}
\begin{itemize}
	\item Ports and Signals are pretty much the same, as far as the usage in a {\em driver} is concerned. Simply carry a literal of the data-type mentioned.
	\item An input port cannot be assigned a value by a concurrent statement. Similarly, an output port cannot serve as an input to another driver. To do so, another port type called {\em buffer} must be used.
	\item A driver event is triggered only when one of the dependencies undergo a change in value, unlike sequential execution of a program.
	\item Each uninitialised signal has a default value corresponding to the the left-most value in description. (\texttt{type bit: \{'0', '1'\}}).
	\item For explicit definition, you can declare it like  \texttt{signal w: bit := '0'}.
	\item VHDL treats each continuous sequence of values of the ports/signals as a waveform. The transactions are piled up in a {\em queue}. The algorithm is as follows.
	\begin{itemize}
		\item 	Once you execute (compile?) the events at $t=0$, you create a bunch of transactions that need to be performed. These are pushed into a T-queue.
		\item The pointer moves to the head of the queue and implements the transactions at that point in time. The resultant driver changes make further additions to the T-queue, and the loop goes on forever.
		\item Delay $\delta$ can be explicitly mentioned. \texttt{a <= not a after 5ns;}
		\item The approach is kind of a breadth-first approach. All the transactions/events at the current point in time executed before moving at a point in time.
	\end{itemize}
	\item The standard way of simulation works only on a fixed value of inputs as defined. What we'd want is to create events on the inputs, to capture the complete picture. For this, we create a {\em testbench}.
	\item Testbench is self-contained, with no entities.
	\begin{itemize}
		\item In the interior, we have the 6 signals that are the entities of the DUT.
		\item To this, we also add a component (which is the DUT), which is defined separately
		\item A port map is defined, which maps the signals of the testbench to the ports of the component.
\end{itemize}		
	\item Comments in the code begin with \texttt{--}.
	\item {\bf Simulators}: \texttt{GHDL} (free), \texttt{ModelSim} (proprietary).
\end{itemize}

\section*{Hands-On}
\begin{itemize}
	\item The \texttt{compile\_ghdl.sh} file provided lists all the files that need to be compiled, which basically includes the main file and testbench.
	\item The testbench reads the file \texttt{TRACEFILE} and writes the output in \texttt{OUTPUTS}. This basically automates all possible test cases and checks whether the simulation is working fine.
	\item For usage instructions on the changes to be made, refer to \texttt{../Generic Testbench/README.txt}
\end{itemize}

\section*{Running The Code}
Using the generic testbench provided for the \texttt{TwoBitAdder}, any circuit of the sort 8-input, 2-output can be synthesised. In this case, the changes required would be:
\begin{itemize}
	\item Let us say the entity is \texttt{component}, declared in \texttt{file.vhd}. Note that the definitions in the testbench and DUT would be of the \texttt{component}, and the filename is only used for the compilation.
	\item The definition of the circuit changes in the \texttt{file.vhd}, which is the real logic definition of the circuit.
	\item The testbench holds for this (if the DUT is used, the Testbench is pretty generic, as the name suggests). However, \texttt{TRACEFILE} needs to be updated, so that verification can be done.
	\item In case of a failure, check the \texttt{OUTPUT} can be used to verify what went wrong.
	\item \texttt{compile\_ghdl.sh} is a wrapper to compile all the required \texttt{.vhd} files. This must include the following: (\texttt{-a} mode is for analysing; \texttt{-m} mode is for )
	\begin{itemize}
		\item The file \texttt{file.vhd}
		\item The generic testbench
		\item DUT declarations, for port mapping
		\item Running the testbench executable?
	\end{itemize}
	\item {\em ... to be continued.}
\end{itemize}

\end{document}